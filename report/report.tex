\documentclass[12pt,a4paper,openright,dvipsnames]{mwart}

\usepackage{lmodern}
\usepackage[T1]{polski}
\usepackage[utf8]{inputenc}

\usepackage[a4paper,
            tmargin=2cm,
            bmargin=2cm,
            lmargin=2cm,
            rmargin=2cm,
            bindingoffset=0cm]{geometry}

\usepackage{tocloft}
\usepackage{hyperref}

\usepackage{amsmath}
\usepackage{amssymb}
\usepackage{siunitx}

\usepackage{listings}

\usepackage{graphicx}
\usepackage{subfig}
\usepackage{float}
\usepackage{booktabs}

\usepackage{xcolor}


\hypersetup{
    colorlinks,
    citecolor=black,
    filecolor=black,
    linkcolor=black,
    urlcolor=black
}


\newcommand{\vtitle}{ESP32 Synthesizer}
\newcommand{\vauthors}{
    Jakub Karbowski
}


\begin{document}
\begin{titlepage}
    \centering

    \textsc{Akademia Górniczo-Hutnicza im. Stanisława Staszica w Krakowie}

    \vspace{\stretch{1}}

    {\LARGE\bfseries \vtitle\\}
    \rule{3in}{0.4pt} \\
    \today

    \vspace{\stretch{1}}

    \large\vauthors

    \vspace*{\stretch{2}}

\end{titlepage}



\section{Wstęp}

Celem projektu było zbudowanie syntezatora cyfrowego na
platformie ESP32. Syntezator ten ma być w stanie odtwarzać
dźwięki na podstawie komend MIDI, które będą przesyłane
do niego przez port szeregowy.
Taki interfejs pozwala na łatwe połączenie z komputerem lub
keyboardem MIDI.


\section{Specyfikacja}

Syntezator ma obsługiwać następujące funkcje:
\begin{enumerate}
    \item dekodowanie pełnego zestawu komend MIDI,
    \item polifonię na przynajmniej 4 głosy,
    \item funkcję ADSR (Attack, Decay, Sustain, Release),
    \item oscylatory piłokształtne,
    \item odtwarzanie dźwięku przez DAC protokołem I2S.
\end{enumerate}


\section{Zbudowana platforma}

Użyte komponenty:
\begin{enumerate}
    \item ESP32 DevKitC,
    \item Adafruit UDA1334A I2S DAC,
    \item głośnik, który znalazłem w śmieciach.
\end{enumerate}






\end{document}
